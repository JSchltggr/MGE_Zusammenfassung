%! Author = joels
%! Date = 05/01/2021

\section{Einführung WPF}
\textbf{WPF:} Windows Presentation Foundation
\subsection{Layout/Grössen}
Layout in C\# oder XAML geschrieben. XAML ist leichter und kürzer.\\
Als Grösseneinheit wird DIP (Device Independent Pixels) verwendet.
\subsection{Hello WPF}
\textbf{\textcolor{blue}{Dateien:}}
\begin{itemize}[topsep=0pt, leftmargin=4mm]
    \setlength\itemsep{-0.3em}
    \item App.xaml: Markup der Startup-Klasse
    \item App.xaml.cs: Coed-Behind der Startup-Klasse
    \item MainWindow.xaml: Markup des Hauptfensters
    \item MainWindow.xaml.cs: Code-Behind des Hauptfensters
    \item AssemblyInfo.cs: Projektspezifische Meta-Daten
\end{itemize}
\textbf{\textcolor{blue}{Deployment:}}
\begin{itemize}[topsep=0pt, leftmargin=4mm]
    \setlength\itemsep{-0.3em}
    \item Framework-Dependent Executable (FDE): .NET Core muss manuell installiert werden. Erzeugt sehr kleines Binary.
    \item Self-Contained Deployment (SCD): .NET Core in Binary integriert.\\
    $\rightarrow$ Sehr grosses Binary (150MB für hello world)
\end{itemize}