%! Author = joels
%! Date = 05/01/2021

\section{Xamarin und Ausblick}
\begin{itemize}[topsep=0pt, leftmargin=4mm]
    \setlength\itemsep{-0.3em}
    \item Gemeinsame Codebasis in C\# / XAML oder F\# / XAML
    \item Zielplattformen: Android, iOS, iPadOS, watchOS, tvOS, macOS
    \item Bei der Kompilierung werden native Apps erzeugt
    \item 100\%ige Verfügbarkeit der nativen APIs in C\#
    \item Benutzung gewohnter .NET-Tools (Visual Studio, NuGet, etc.)
\end{itemize}
\subsection{Referenzarchitektur}
\textbf{\textcolor{blue}{Shared Code:}}
\begin{itemize}[topsep=0pt, leftmargin=4mm]
    \setlength\itemsep{-0.3em}
    \item Von allen Plattformen (IOS, Android, .NET) geteilt
    \item Idealerweise möglichst gross
    \item Interfaces zur Abstraktion von Plattform Details
    \item .NET Standard Projekt
\end{itemize}
\textbf{\textcolor{blue}{Platform Code:}}
\begin{itemize}[topsep=0pt, leftmargin=4mm]
    \setlength\itemsep{-0.3em}
    \item Ein Projekt pro Ziel-Plattform
    \item Idealerweise möglichst klein
    \item Implementierung der Plattform-Interfaces
    \item Projekttyp abhängig von Ziel-Plattform
\end{itemize}
\subsection{Xamarin.Essentials}
\begin{itemize}[topsep=0pt, leftmargin=4mm]
    \setlength\itemsep{-0.3em}
    \item Sammlung an Platform Services
    \begin{itemize}[topsep=0pt, leftmargin=4mm]
        \setlength\itemsep{-0.3em}
        \item Sensoren: Batterie, Kompass, ...
        \item Schnittstellen: Berechtigungen, Telefon, ...
        \item Utilities: Threading, Umrechnungen, ...
    \end{itemize}
    \item Spart viel Zeit und Nerven
    \begin{itemize}[topsep=0pt, leftmargin=4mm]
        \setlength\itemsep{-0.3em}
        \item Eine Schnittstelle für alle Plattformen
        \item Tipp: Trotzdem hinter Interface abstrahieren
    \end{itemize}
    \item Integration via NuGet
\end{itemize}
\subsection{Xamarin Traditional}
\begin{itemize}[topsep=0pt, leftmargin=4mm]
    \setlength\itemsep{-0.3em}
    \item Definition des UI pro Zielplattform
    \begin{itemize}[topsep=0pt, leftmargin=4mm]
        \setlength\itemsep{-0.3em}
        \item Verwendung der nativen Konzepte
        \item Android: XML, Activities, Fragmente, ...
    \end{itemize}
    \item Vorteile:
    \begin{itemize}[topsep=0pt, leftmargin=4mm]
        \setlength\itemsep{-0.3em}
        \item Performance
        \item Voller Funktionsumfang der Zielplattform
        \item Portierbarkeit bestehender Apps
    \end{itemize}
    \item Nachteile:
    \begin{itemize}[topsep=0pt, leftmargin=4mm]
        \setlength\itemsep{-0.3em}
        \item Mehrfache Implementierung des UI
        \item Viele unterschiedliche Technologien
    \end{itemize}
\end{itemize}
\subsection{Xamarin.Forms}
\begin{itemize}[topsep=0pt, leftmargin=4mm]
    \setlength\itemsep{-0.3em}
    \item Definition des UI im Shared Code
    \begin{itemize}[topsep=0pt, leftmargin=4mm]
        \setlength\itemsep{-0.3em}
        \item Verwendung von XAML
        \item Tipp: Eigenes Projekt für Xamarin.Forms
    \end{itemize}
    \item Vorteile:
    \begin{itemize}[topsep=0pt, leftmargin=4mm]
        \setlength\itemsep{-0.3em}
        \item UI muss nur einmalig implementiert werden
        \item Weniger Technologien
    \end{itemize}
    \item Nachteile:
    \begin{itemize}[topsep=0pt, leftmargin=4mm]
        \setlength\itemsep{-0.3em}
        \item Einschränkungen bei UI Gestaltung
        \item Leichte Einbussen bei Performance
        \item Schwierige Portierbarkeit bestehender Apps
    \end{itemize}
\end{itemize}
\subsubsection{Xamarin.Forms – Renderer}
\begin{itemize}[topsep=0pt, leftmargin=4mm]
    \setlength\itemsep{-0.3em}
    \item Xamarin.Forms enthält diverse Controls. Bsp: Button
    \item Renderer-Klassen erledigen das Mapping von XAML-Controls auf native Controls. Ist Bestandteil von Xamarin.Forms. Bsp: ButtonRenderer für Android
    \item Anpassungsmöglichkeiten: Styling, Eigene Renderer, Eigene XAML-Controls
\end{itemize}
\subsubsection{Xamarin.Forms – Vergleich zu WPF}
\textbf{\textcolor{blue}{Gemeinsamkeiten:}}
\begin{itemize}[topsep=0pt, leftmargin=4mm]
    \setlength\itemsep{-0.3em}
    \item Aufteilung in XAML und Code Behind
    \item Application-Klasse
    \item Resources und Styles
    \item Markup Extensions
    \item Data Binding
    \item Commands
    \item Value Converter
\end{itemize}
\textbf{\textcolor{blue}{Unterschiede:}}
\begin{itemize}[topsep=0pt, leftmargin=4mm]
    \setlength\itemsep{-0.3em}
    \item Control Libraries
    \item Anzahl UI-Projekte
    \item XAML Dialekt
    \begin{itemize}[topsep=0pt, leftmargin=4mm]
        \setlength\itemsep{-0.3em}
        \item MainPage statt MainWindow
        \item BindingContext statt DataContext
        \item IsVisible statt Visibility
        \item Margin mit Komma statt Spaces
    \end{itemize}
    \item Hilfsklassen in Xamarin.Forms
    \begin{itemize}[topsep=0pt, leftmargin=4mm]
        \setlength\itemsep{-0.3em}
        \item Service Locator (Dependency Injection)
        \item Navigation Service
    \end{itemize}
\end{itemize}